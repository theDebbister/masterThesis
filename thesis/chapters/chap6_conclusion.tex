\newchap{Conclusion}
\label{chap:6_conclusion}
In the beginning I listed a few questions that I would like to answer while working on this thesis.

\begin{enumerate}
\item What types of phonetic data is available and how can it be used?
\item Which computational models can be used to create phonetic transcriptions?
\item How can we use phonetic features to create phonetic transcriptions?
\item Is there any significant difference in comparing spoken or written languages?
\item Does written text represent language well enough to justify text-based research only?
\end{enumerate}


\section{Future work}
Non surprisingly, apart from many exciting things I \textit{could} do, there are many others that would have gone beyond the scope of this thesis. I would like to list a few entry points on where further research could start. 

\begin{itemize}
\item \textbf{Data preprocessing}: \cite{Ashby&Bartley.2021} cleaned broad transcriptions for Bulgarian and replaced allophones by their standard phoneme. This could further improve model quality by having consistent broad transcriptions.
\item 
\end{itemize}
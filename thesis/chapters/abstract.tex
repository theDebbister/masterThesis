\newpage
\phantomsection % to get the hyperlinks (and bookmarks in PDF) right for index, list of files, bibliography, etc.
\addcontentsline{toc}{chapter}{Abstract}



\begin{abstract}
\section*{Abstract}
In this master's thesis I tackle the task of multilingual \acf{g2p} conversion. In a first part, I familiarized myself with the current data situation. Most \acs{g2p} data is available as word lists that contain mappings from graphemes to phonemes. There exists a dataset containing full text transcriptions of a well-known short story called \acf{nws}, which I used to test my \acs{g2p} models. I created models for 22 different languages using an existing transformer model. As an additional contribution I incorporated phonetic features into the input to the models by extending the input phonemes with encoded phonetic features. The models I trained could keep up with state-of-the-art G2P models and outperformed existing models for some languages. I tested the models on different datasets and the results of the models diverged greatly ont he different datasets. This lead me to the conclusion that phonetic transcriptions are not as standardized as writing systems used for written language. Removing tones from the datasets in the preprocessing step lead to a great improvement of the results for some languages while other preprocessing steps for other languages did not improve the results at all. The models I created that received additional phonetic features as input did not outperform the state-of-the-art models. However, this is most likely due to the way I encoded the features and other strategies to encode the features could be employed in future research. As for many of my languages there is no result available, my models can serve as a baseline for future research. 

All scripts and files produced along with this thesis are found in my GitHub repository: \url{https://github.com/theDebbister/masterThesis}.
\end{abstract}

\begin{abstract}
\selectlanguage{ngerman}
\section*{Zusammenfassung}
In dieser Masterarbeit beschäftige ich mich mit der Aufgabe der mehrsprachigen \acf{g2p} Konvertierung. In einem ersten Teil habe ich mich mit der aktuellen Datenlage vertraut gemacht. Die meisten \acs{g2p}-Daten sind als Wortlisten verfügbar, die Zuordnungen von Graphemen zu Phonemen enthalten. Es gibt einen Datensatz mit Volltexttranskriptionen einer bekannten Kurzgeschichte namens \textit{Der Nordwind und die Sonne} \acs{nws}, den ich zum Testen meiner \acs{g2p}-Modelle verwendet habe. Ich habe Modelle für 22 verschiedene Sprachen erstellt, wofür ich ein bestehendes Transformer-Modell verwendet habe. Als zusätzlichen Beitrag habe ich phonetische Merkmale in die Inputs der Modelle integriert, indem ich die Input-Phoneme um kodierte phonetische Merkmale erweitert habe. Die von mir trainierten Modelle konnten mit den modernsten G2P-Modellen mithalten und übertrafen die bestehenden Modelle für einige Sprachen. Ich testete die Modelle auf verschiedenen Datensätzen und die Ergebnisse der Modelle wichen auf den verschiedenen Datensätzen stark voneinander ab. Dies führte mich zu der Schlussfolgerung, dass phonetische Transkriptionen nicht so standardisiert sind wie Schriftsysteme, die für geschriebene Sprache verwendet werden. Das Entfernen von lexikalischen Tönen aus den Datensätzen führte bei einigen Sprachen zu einer großen Verbesserung der Ergebnisse, während andere Schritte zur Aufbereitung der Daten bei anderen Sprachen die Ergebnisse überhaupt nicht verbesserten. Die von mir erstellten Modelle, die zusätzliche phonetische Merkmale als Input erhielten, übertrafen die state-of-the-art-Modelle nicht. Dies ist jedoch höchstwahrscheinlich auf die Art und Weise zurückzuführen, wie ich die Merkmale kodiert habe, und andere Strategien zur Kodierung der Merkmale könnten in zukünftigen Forschungsarbeiten eingesetzt werden. Da für viele meiner Sprachen noch keine Ergebnisse vorliegen, können meine Modelle als Grundlage für künftige Forschungen dienen. 

Alle Scripts und Dokumente, die ich im Zusammenhang mit dieser Arbeit erarbeitete, sind auf meinem GitHub zu finden: \url{https://github.com/theDebbister/masterThesis}.


\selectlanguage{english}
\end{abstract}
\newpage

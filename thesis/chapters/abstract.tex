\newpage
\phantomsection % to get the hyperlinks (and bookmarks in PDF) right for index, list of files, bibliography, etc.
\addcontentsline{toc}{chapter}{Abstract}



\begin{abstract}
\section*{Abstract}
In this master's thesis I tackle the task of multilingual grapheme-to-phoneme conversion. In a first part, I familiarized myself with the current data situation. Most \acl{g2p} data is available as word lists that contain mappings from graphemes to phonemes. There exists another dataset containing full text transcriptions of a well-known short story called \acl{nws}, which I used as well. I created models for 22 different languages using an existing transformer model. As an additional challenge I tried to incorporate phonetic features by extending the input phonemes with encoded phonetic features. The models I trained could keep up with state-of-the-art G2P models and outperformed existing models for some languages. Phonetic transcriptions are not as standardized as writing systems used for written language. I tested the models on different datasets and the results of the models diverged greatly. As for many of my languages there is no result available, my models can serve as a baseline for future research. 

All scripts and files produced along with this thesis are found in my GitHub repository: \url{https://github.com/theDebbister/masterThesis}.
\end{abstract}

\begin{abstract}
\selectlanguage{ngerman}
\section*{Zusammenfassung}
In dieser Masterarbeit beschäftige ich mich mit der Aufgabe der mehrsprachigen grapheme-to-phoneme Konvertierung (G2P). In einem ersten Teil habe ich mich mit der aktuellen Datenlage vertraut gemacht. Die meisten G2P-Daten sind als Wortlisten verfügbar, die Zuordnungen von Graphemen zu Phonemen enthalten. Es gibt einen weiteren Datensatz mit Volltexttranskriptionen einer bekannten Kurzgeschichte namens \textit{Der Nordwind und die Sonne}, den ich ebenfalls verwendet habe. Ich habe Modelle für 22 verschiedene Sprachen erstellt und dabei ein bestehendes transformer Modell verwendet. Als zusätzliche Herausforderung habe ich versucht, phonetische Features einzubeziehen, indem ich die Phoneme mit kodierten phonetischen Features erweitert habe. Die von mir trainierten Modelle konnten mit den modernsten G2P-Modellen mithalten und übertrafen die bestehenden Modelle für einige Sprachen. Phonetische Transkriptionen sind nicht so standardisiert wie Schriftsysteme, die für geschriebene Sprache verwendet werden. Ich habe die Modelle an verschiedenen Datensätzen getestet und die Ergebnisse der Modelle wichen stark voneinander ab. Da für viele meiner Sprachen noch keine Ergebnisse vorliegen, können meine Modelle als Grundlage für künftige Forschungen dienen.

Alle Scripts und Dokumente, die ich im Zusammenhang mit dieser Arbeit erarbeitete, sind auf meinem GitHub zu finden: \url{https://github.com/theDebbister/masterThesis}.


\selectlanguage{english}
\end{abstract}
\newpage

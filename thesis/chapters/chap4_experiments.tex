
\newchap{Experiments}
\label{chap:4_exp}
This chapter presents the experiments that I conducted for this thesis. The previous chapters listed the different steps and problems that arise when trying to create and analyse a phonetic corpus. 

\section{Typewriting pdf phonetic transcriptions and training a model to do OCR}


\section{Creating full texts out of pronunciation dictionaries}
In order to create full texts out of pronunciation dictionaries, I created a simple python script. There were several problems that needed to be addressed in order to create those texts. 

\begin{itemize}
\item The pronunciation dictionaries sometimes included duplicates with different pronunciations. This is not surprising but still it needs to be handled well. A possible solution to this is to include \ac{pos} tags. Although this is generally possible it would  mean an significantly greater effort which might exceed this thesis' scope. 
\item Not all words are in the dictionaries. This problematic as the texts cannot be fully transcribe in this way. A solution would be to transcribe only sentences and then pick those sentences that are fully transcribed. Another solution could be to have statistics of the most frequent missing words (in English 'and' is missing) and either transcribe them manually if possible, find transcriptions in the internet or check in the narrow / broad pronunciation dictionary and use this word instead. The latter possibility might corrupt the data, but as there are not that many missing words, it is worth a try.
Another way to deal with missing words is to iteratively split the words in two or possibly three or more parts and check whether the subwords are in the dictionary. This depends on the specific language. In the case of English it was necessary to exclude splitting of the first and the last character separately as those are in the dictionary but with their alphabet reciting pronunciation which is typically not used within a word. For this approach it is necessary to have some language specific expertise.
\item The IPA allows to transcribe intonation segments. In German, those correspond mostly to punctuation like end of sentence symbols or commas. But this must not be true for every case. In order to include those a close examination of the 
\end{itemize}

There are several decisions to be made. 

\section{Automatic \ac{g2p}}
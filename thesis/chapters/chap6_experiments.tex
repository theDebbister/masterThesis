
\newchap{Experiments: Automatic Grapheme-to-Phoneme Conversion}
\label{chap:exp}
In this chapter, I present the experiments I conducted to obtain computational models to create phonetic transcriptions. The datasets I use to train the models are presented in section \ref{sec:dataset-models}. The model that I am using for my \ac{g2p} experiments is explained in section \ref{sec:cmu}. In order to compare my results to already existing models, I will use the results of the \ac{sigm} tasks 2020 and 2021. They present results for quite a few languages. Also, they use the same data type as I do. I introduced some of those models in chapter \ref{chap:tech-background}. 

\subparagraph{Reproducability}
All scripts that I used to preprocess, train and evaluate my models are found on my GitHub\myfootnote{\url{https://github.com/theDebbister/masterThesis}}. I installed the model on my machine as specified on the model's GitHub\myfootnote{\url{https://github.com/cmusphinx/g2p-seq2seq}}. 

\section{Datasets}
\label{sec:dataset-models}
As a result of my first practical part in chapter \ref{chap:data_collection}, I presented two datasets that I will now use to perform the experiments in this chapter. The WikiPron corpus serves as a training dataset for my model. Note that WikiPron continuously adds data to their repository. This means that there might be new languages that are in the 100 language corpus and that I did not use. Additionally, there were languages which had a WikiPron dictionary that was smaller than 1,000 grapheme-phoneme pairs. \citet{Ashby-Bartley.2021} have found that a dataset with 1,000 samples is not enough to train a decent model for \ac{g2p} conversion which is why I excluded datasets for languages with less than 1,000 samples. Their finding is relevant for my thesis as they based their conclusion on similar WikiPron data. For some languages in the WikiPron data there is a cleaned version available. Whenever a filtered version is available I used the filtered version as the basis for my experiments.

The \ac{nws} corpus is much smaller which is why I will use it as a test dataset only. As the \ac{nws} corpus is quite well known among linguists or at least among phoneticians, it will hopefully give interesting insights when testing the models on these short texts. 

Table \ref{tab:all_langs} shows all languages that I am using in my experiments. Generally, broad and narrow transcriptions are treated as separate languages and thus trained separately as well. The same is true for dialects if there is any information available. For example: for American and British English I will train different models. 

\section{CMUSphinx}
\label{sec:cmu}
For my personal experiments, I decided to use the CMUSphinx \ac{s2s} \ac{g2p} model. This model has been used in the \ac{sigm} 2021 task as part of the Dialpad ensemble \citep{gautam.2021}. It was not used on many languages but promised a good performance which is why I decided to use this model for this present thesis. The CMUSphinx model is a transformer-based \ac{s2s} model implemented with tensorflow. Unfortunatley, there is not a lot of information available online about the model. There exists a pre-trained version of the model for \ac{g2p}. However, they use a transcription format other than \ac{ipalpha} which means it cannot be used for my dataset\myfootnote{\url{https://github.com/cmusphinx/g2p-seq2seq}}. 

For all my experiments, the following two settings of the model are the same:
\begin{itemize}
\item The CMU model splits the data automatically if not specified otherwise. The automatic split is 85\% training data, 5\% development data and 10\% test data. I did not change this split.
\item The hyperparameters are left as default if not mentioned otherwise. This means that I did not adapt the model and used it as it can be found online. More on the default settings can be found on the model's GitHub.
\end{itemize}

The model takes as input parallel examples of grapheme and phoneme sequences. This is typically a tsv (tab-separated-values) file structured as shown in table \ref{tab:example-wikipron}. Sometimes this data format is referred to as a input dictionary.

\section{Training settings}
\label{sec:train-settings}
There are different settings which I will use to train the models and analyse their performance. The settings are designed in a way that each setting introduces a new or more complex adaption of the previous setting. The first experiments are set up very simply and without much effort. Then I continuously add more complexity. Below I added a short description of each setting as an overview. I will add more details on the individual steps in separate sections. 

\subparagraph{Setting 1: Baseline short}
This is the most basic setting. I will train a model for each language to get a baseline result. The model is trained with the least amount of effort. The model is trained for the minimum number of steps which is 10,000 steps in this case. 

\subparagraph{Setting 2: Baseline long}
This setting is similar to setting 1 except that the model is trained as long as possible for each language. All models have been trained for 200,000 steps. This setting is supposed to show if training for a considerably longer time changes the results a lot or if it does not result in a better performance.

\subparagraph{Setting 3: Baseline clean short}
I will train another model that is the same like the baseline short, but I will use the cleaned WikiPron data. What I will clean is described below (section \ref{preprocess}).

\subparagraph{Setting 4: Baseline clean long}
The same as setting 3 except that the models were trained for as long as possible like in setting 2.

\subparagraph{Setting 5: Feature input version 1 short}
The final experiments will be with phonetic features as input. I used the cleaned WikiPron data from setting 3 and added phonetic features to it. In section \ref{sec:feature_enc}, I explain how I encode the features. 

\subparagraph{Setting 6: Feature input version 1 long}
The same as setting 5, but again the models are trained for 200,000 steps like in setting 2.

\subparagraph{Setting 7: Feature input version 2 short}
This setting is the same as setting 5, but with a different version of the input feature data.

\subparagraph{Setting 8: Feature input version 2 long}
The same as setting 7, but again the models are trained for 200,000 steps like in setting 2.


The training time for the short model for one language with 10,000 steps is about 20 minutes. The long models took another six hours to complete the training per language.

\section{Preprocessing}
\label{preprocess}
In section \ref{sec:ipa}, I wrote about the incompleteness and difficulties of transcribing using the \ac{ipalpha}. How exactly a sound is mapped to an \ac{ipalpha} symbol also depends on whoever transcribes a particular text. The WikiPron data has been put together by many different people. There are conventions on how to add transcriptions to Wiktionary\myfootnote{\url{https://en.wiktionary.org/wiki/Wiktionary:Pronunciation}}, but there still might be inconsistencies. Other than that, it is always possible that a transcription is correct, but computational models just cannot handle it well. That said I will carefully examine and clean the datasets. In chapter \ref{chap:mwl}, example \ref{ex:mean-word-preprocess} I have already given an example of the type of questions that need to be answered to preprocess a text to perform an analysis on that text. This current section is centered around a multiple of such questions.
Some of the preprocessing will be done for both of my dataset to be consistent in the preprocessing. As the datasets are very different there needs to be done individual preprocessing for each dataset as well.

The first analysis to decide on what to preprocess is a character based comparison of the datasets and the PHOIBLE database. This means that I used the Python segments library\myfootnote{\url{https://pypi.org/project/segments/}} to split all the phonemes into Unicode characters. Then I compared those characters to the PHOIBLE database. As I have explained in section \ref{sec:phoible}, in can use the PHOIBLE database to find phonemes that are uncommon for a specific language. While it is possible that a correctly used phoneme is not in PHOIBLE, it still gives a good overview of uncommon phonemes in the transcription or even points out mistakes in my data. This character based cleaning is done for both datasets. A more detailed list of what needs to be cleaned is found in table \ref{tab:preprocessing}.

There are two more character types not found in table \ref{tab:preprocessing} that I removed from the datasets. The first is \textbf{tones}. The reason for removing tones is that it is not possible to infer tones from graphemes. Although they are used to distinguish meaning, they are not written down, but usually just known. This means that there are grapheme sequences that look exactly the same and their transcription is the same as well \textit{except} for the tones. Example \ref{ex:chinese-tones} shows the same Chinese grapheme together with its transcription. The superscript number represent the tones. For none of the three examples are the tones the same. For the exact same Chinese grapheme there exist three different possible pronunciations. But if only shown the grapheme, there is absolutely no way to find out which tones are meant:

\begin{covsubexamples}
\label{ex:chinese-tones}
\item 
\begin{CJK*}{UTF8}{bsmi}
浸 \>\>\>\>\textipa{\t{tC}\super h i n} \textsuperscript{2}\textsuperscript{1}\textsuperscript{4}
\end{CJK*}
\item 
\begin{CJK*}{UTF8}{bsmi}
浸 \>\>\>\>\textipa{\t{tC}\super h i n} \textsuperscript{5}\textsuperscript{1}
\end{CJK*}
\item  
\begin{CJK*}{UTF8}{bsmi}
浸 \>\>\>\>\textipa{\t{tC}\super h i n} \textsuperscript{5}\textsuperscript{5}
\end{CJK*}
\end{covsubexamples}

For the reason of the ambiguity shown in example \ref{ex:chinese-tones}, I decided to exclude tones. There are different ways to represent tones and I excluded all of them (refer to figure \ref{fig:ipa_chart} for the different tone versions). 

The second additional preprocessing step I performed is to remove all \textbf{punctuation symbols}. This accounts mostly for the orthographic version of the \ac{nws} short stories. For character-based modelling, punctuation does not add any valuable information as this only becomes relevant on a sentence basis. 

In example \ref{ex:cleaning} I list a few phonetic transcriptions before and after cleaning. 

\begin{covsubexamples}[preamble={This example shows strings from the WikiPron data before and after cleaning them. The left-hand side of the arrows shows the uncleaned version and the right-hand side the cleaned version. I added the language code and the transcription type and the original grapheme sequence in parentheses.}]
\label{ex:cleaning}
\item deu, narrow

\textipa{a: b @ n t m a: l \t{ts} a \textsubarch{I} t} $\rightarrow$ \textipa{a: b @ n t m a: l t s a \textsubarch{I} t} (Abendmahlzeit)
\item fin, broad

\textipa{A: l: o t A\super x}  $\rightarrow$ \textipa{A: l: o t A P} (aallota)
\item cmn, broad

\textipa{\t{tC}\super h i n} \textsuperscript{2}\textsuperscript{1}\textsuperscript{4} $\rightarrow$ \textipa{t C \super h i n}
\begin{CJK*}{UTF8}{bsmi}
(浸)
\end{CJK*}
\item eng uk, narrow

\textipa{@ l {3\textrhoticity} d Z I k} $\rightarrow$ \textipa{@ l {@\textrhoticity} d Z I k} (allergic)
\end{covsubexamples}

\tab{tab:preprocessing}{The table shows what phonemes where changed or excluded and what the reason is for this preprocessing. All characters that were excluded are replaced by a NULL value. }{
\begin{tabularx}{1.05\textwidth}{
| 	>{\raggedright\arraybackslash}l | 
	>{\raggedright\arraybackslash\hsize=.5\hsize\linewidth=\hsize}X | 
	>{\raggedright\arraybackslash}l  | 
	>{\raggedright\arraybackslash\hsize=1.5\hsize\linewidth=\hsize}X |}
\hline
\textbf{Phon.} & \textbf{Unicode name} & \textbf{Repl.} & \textbf{Explanation} \\
\hline
\hline
 \textipa{"} 					& \scriptsize{MODIFIER LETTER VERTICAL LINE} 				& NULL 						& \multirow[t]{6}{\hsize}{These are all \ac{ipalpha} suprasegmentals except the long and half long marker and the extra short (\textipa{: ; \u{}}). The reason why these were excluded is that they are not meaningful on the character level. The vertical lines, for example, mark intonation groups which only matter in a larger sentence or text context. There are a few rare occurrences of COMBINING VERTICAL LINE ABOVE which is probably meant to be MODIFIER LETTER VERTICAL LINE as they look similar. It is excluded as well.} \\
 
\textipa{""} 					& \scriptsize{MODIFIER LETTER LOW VERTICAL LINE}			& NULL						&  \\
\textipa{\textvertline} 		& \scriptsize{VERTICAL LINE} 								& NULL						&  \\
\textipa{\textdoublevertline} 	& \scriptsize{DOUBLE VERTICAL LINE} 						& NULL 						&  \\
\textipa{.} 					& \scriptsize{FULL STOP} 									& NULL 						&  \\
\textipa{\t*{}}					& \scriptsize{UNDERTIE} 									& NULL 						&  \\\hline
\textipa{\t{}}					& \scriptsize{COMBINING DOUBLE INVERTED BREVE} 				& NULL						&  \multirow[t]{2}{\hsize}{Both tie bars below and above are excluded in PHOIBLE\myfootnote{https://phoible.org/conventions} which is why I am excluding it as well. Put plainly, those do not add any additional information that cannot be derived otherwise.} \\
\textipa{\t*{}}					& \scriptsize{COMBINING DOUBLE BREVE BELOW} 				& NULL						&  \\\hline
\textipa{3\textrhoticity} 		& \scriptsize{LATIN SMALL LETTER REVERSED OPEN E WITH HOOK} & \textipa{@\textrhoticity} &  Both base letters are very similar vowels. It is just more common to use the latter than the former. \\\hline
g						 		& \scriptsize{LATIN SMALL LETTER G} 						& \textipa{g} 				&  The \ac{ipalpha} `\textipa{g}' has a different code point and is a different character than the typical keyboard small Latin `g'. This is just an \ac{ipalpha} decision. For some fonts the two characters do not look different, for some they do.  \\\hline
$\sim$							& \scriptsize{SWUNG DASH} 									& NULL 						& \multirow[t]{4}{\hsize}{All of these characters make out less than 1\% of their respective dataset, most of the time it is less than 0.1\%. A close examination of the dataset and the Wiktionary transcription conventions for the respective language did not show any reason why to keep the phonemes. Note that the `v' for the tilde is only there for correct representation.}  \\
,							 	& \scriptsize{COMMA} 										& NULL 						&  \\
\textipa{\~v} 					& \scriptsize{TILDE} 										& NULL 						& \\
\textsuperscript{\textipa{@}}	& \scriptsize{MODIFIER LETTER SMALL SCHWA} 					& NULL &  \\\hline
\textsuperscript{x}				& \scriptsize{MODIFIER LETTER SMALL X} 						& \textipa{P}				&  The \textsuperscript{x} only occurred in the broad Finnish transcription and is used to denote possible gemination. In the narrow transcriptions there is a glottal stop instead. The occurrence of glottal stops and gemination follows the same rules. Therefore, for consistency, the gemination \textsuperscript{x} is mapped to a LATIN LETTER GLOTTAL STOP.   \\
\hline
( ) 						& \scriptsize{( SUPERSCRIPT ) [ LEFT | RIGHT ] PARENTHESIS}		& NULL							& Parentheses are used to denote optionality for phonemes or tones. WikiPron actually discards those but keeps the content\myfootnote{\url{https://github.com/CUNY-CL/wikipron}}. I will do the same for all parenthesis found.  \\\hline
\end{tabularx}}{Preprocessing}

\subsection*{NWS corpus}
I had to transform the \ac{nws} stories into dictionary format in order to be able to use them as testing data. It was necessary to tokenize both orthographic and phonetic texts and then align orthographic tokens and phonetic tokens. As the number of tokens (when split naively at white space) is not always the same for orthographic and phonetic text, it was necessary to align the orthographic and phonetic texts for some languages manually. While this is not a problem for languages that I know how to pronounce, it is a bit tricky for languages completely unknown to me. Luckily there are tools online that provide a rough pronunciation of a word in a given language or even a phonetic transcription (although rarely in \ac{ipalpha})\myfootnote{For most languages I could use Google translator (\url{https://translate.google.com/?hl=de&sl=ko&tl=de&op=translate}), for languages like Hebrew, I could ask someone who knows the language to help me out.}. For my experiments described in section \ref{sec:coverage} and chapter \ref{chap:mwl} I had to tokenize both phonetic and orthographic texts for most of my languages. Please refer to the before mentioned section and chapter to find more details on tokenization. 

Apart form the characters listed in table \ref{tab:preprocessing} that I excluded from the \ac{nws} texts, there were a few more characters that needed to be excluded. 
As the models cannot handle characters that are not in their vocabulary, I needed to remove the unknown characters to be able to evaluate the models on the \ac{nws} stories. The vocabulary of the neural models I use contains all characters that were in the training data. The models I train consequently contain all characters that are in the training split of the WikiPron data. However, for some characters, they were in the vocabulary, but the input data was in the wrong Unicode normalization form as described in section \ref{sec:unicode_ipa}. For example, some characters were precomposed characters in the WikiPron training data while they were not precomposed in the \ac{nws} stories I used to test the models. Both characters in the model's vocabulary and the \ac{nws} stories look exactly the same but the model treats them as two completely different characters because they have a different code point. I could easily solve that problem by converting the \ac{nws} short stories to the same normalization form as the model's vocabulary. See example \ref{ex:precomposed} where I illustrate the difference between precomposed and non-precomposed characters.


\subsection*{WikiPron} 
Just as for the \ac{nws} corpus, I needed to do some additional preprocessing for the WikiPron data. As the CMU model does not expect the orthographic input sequences to contain white space, I replaced any white space in the input grapheme sequences with underscores. This was necessary only for Vietnamese. In a first run, I did not replace white space by underscores. When running the evaluation on the Vietnamese model, it did not work but showed an error message. As the model creates a vocabulary file, I had a look at it which revealed the following: 
\begin{itemize}
\item The model does not actually split the input file at tab characters, but splits it at spaces and then uses the first item of the resulting list as input grapheme and the rest as a list containing the output phoneme segments. Consequently, if the input grapheme contains a white space, everything following it will be in the phoneme part. Part of the input will be treated as a phoneme segment in the training which results in a huge vocabulary and wrong predictions.
\end{itemize}

In Japanese there is the superscript phoneme /$^\beta$/. This phoneme actually means compression, which is a special type of rounding\myfootnote{\url{https://en.wikipedia.org/wiki/Roundedness}}. This special case of rounding in Japanese is not reflected in PHOIBLE nor in the official \ac{ipalpha} but this does not mean it is wrong. As it is quite common in the Japanese data, I decided to keep the phoneme /$^\beta$/. This is one of those cases where the PHOBILE data helped to find a uncommon transcription that is not necessarily wrong.

A last thing I did to clean the WikiPron datasets was to exclude duplicate grapheme sequences with different pronunciations. Although ambiguities and multiple possible pronunciations for one word are linguistically speaking very common, it is not possible for a neural model to distinguish such cases without any context. As \ac{g2p} modelling happens on character basis and not on word basis there is no context available in our case that could account for at least some ambiguities. Also, in the WikiPron data that was preprocessed for the \ac{sigm} task, duplicates were excluded as well. It makes sense to clean all datasets in a similar way to allow for better comparability.


\section{Feature encoding}
\label{sec:feature_enc}
In order to incorporate the phonetic features into the dataset, we decided to add what we refer to as `flags' to the phonemes. Generally, the idea is to encode the phonetic features that PHOIBLE provides for each phoneme and add them to the WikiPron dataset. I will list the steps of my experiments in the following. For all feature experiments I used the WikiPron data that I had cleaned according to what is explained in section \ref{preprocess}.

\subsection*{Version 1}

The first idea to incorporate the phonetic features into the datasets is to use the PHOIBLE feature set for each of the phonemes and encode the information that is found in those features. As there are 37 features for each phoneme they allow to encode much more and more finegrained information than one single character can. By encoding the phonetic features we can add more information that the model can use to learn how graphemes are mapped to phonemes. For example: in some languages, there are vowel-consonant patterns. In English a word typically needs at least one vowel and it is uncommon to have consonant sequences longer than three letters. If it was possible to explicitly encode for each phoneme if it was a vowel or a consonant, the model could abstract that information more easily an learn how sound patterns work. 

In order to add phonetic information to the datasets, I performed the following steps:
\begin{description}
    \item[\textsc{1. step}] I split the list of all 37 PHOIBLE features (see section \ref{sec:phoible}) into two sets of PHOIBLE features. After this step, I had two disjoint sets of features. One containing the first half of all features, the other one containing the second half of all features.
    \item[\textsc{2. step}] For each of the two sets, I used two different capital letters to encode all possible feature combinations that can be obtained by combining all features in the set with all other features in the same set. If there had been two features in the set, `syllabic' and `consonantal', there would have been nine possible combinations. To each of these combinations I assigned one string of two capital letters. Table \ref{tab:feature-encoding} shows how this step looks like for a small example. 
    \item[\textsc{3. step}] Once I completed the above step, I had two sets of strings that I could combine to encode all possible combinations of features for each of the phonemes. Each string represents a different feature combination of one of the two subsets of the features.
    \item[\textsc{4. step}] Then, I could assign two strings to each phoneme in my data. The first string represents the first half of the phoneme's features, the second string represents the second half of the phoneme's features.
\end{description}

    \tab{tab:feature-encoding}{This table shows an example of how the feature encoding works. In this case there are only two features involved. Nonetheless, the procedure works the same for more features. Refer to section \ref{sec:phoible} to understand the meaning of the values for each feature.}{
    \begin{tabular}{|ll|l|}
    \hline
        \textbf{syllabic} & \textbf{consonantal} & \textbf{Feature encoding} \\\hline\hline
        +        & -           & AA       \\
        +        & 0           & AB       \\
        +        & +           & AC       \\
        -        & -           & AD       \\
        -        & 0           & BA       \\
        -        & +           & BB       \\
        0        & -           & BC       \\
        0        & 0           & BD       \\
        0        & +           & CA       \\\hline     
        \end{tabular}
        }{Feature encoding}
After completing the above steps, I got an output that looks similar to the examples below (examples \ref{ex1} and \ref{ex2}). The combination of the two flags is unique for each phoneme.  The model will interpret each two-letter feature string as one phoneme segment.

\begin{covexamples}
\item \label{ex1} \textipa{p} AB CD \textipa{k\super h} AA BC
\item \label{ex2} \textipa{k\super h} AA BC  \ipa{U} BF CC
\end{covexamples}

A problem with this approach is that the sequences get really long if all phoneme segments are now represented by three segments.  Examples \ref{ex1} and \ref{ex2} show that the phoneme sequence gets three times longer. Originally there were only two segments separated by a white space while after the feature flags are added there are six segments. However, the CMU model separates the  phoneme sequences at white spaces and cuts off every segment after 30 segments. This means that only phoneme sequences that are no longer than 10 segments will be represented in full when for each phoneme two additional feature flags are added as in the example above. Example \ref{ex:cut-off} shows how the CMU model cuts off the phoneme sequences.

\begin{covsubexamples}[preamble={The CMU model cuts off phoneme sequences that are longer than 30 segments. Sub-example (a) shows the original phoneme sequence without features and the corresponding grapheme sequence. Sub-example (b) shows the full length phoneme sequence with features and sub-example (c) shows the shortened version with features. It is easily understandable that there is a lot of information lost if a sequence is shortened like that.}]
\label{ex:cut-off}
\item \label{ex:cut-off1} \textipa{y: b U N s b u: x z a \|)I t @ n} \\ Übungsbuchseiten
\item \label{ex:cut-off2} \textipa{y:} BA EF \textipa{b } AH FG \textipa{U } AN GE \textipa{N } AG EC \textipa{s } AJ FE \textipa{b } AH EE \textipa{u: } BA FF \textipa{x } AJ GH \textipa{z } DE HG \textipa{a} BG FG \textipa{{\textsubarch{I}}} DC EH \ipa{t } AB HH \ipa{@ } AA FE \ipa{n} BB EE
\item \label{ex:cut-off3} \textipa{y:} BA EF \textipa{b } AH FG \textipa{U } AN GE \textipa{N } AG EC \textipa{s } AJ FE \textipa{b } AH EE \textipa{u: } BA FF \textipa{x } AJ GH \ipa{z } DE HG \ipa{a} BG FG 
\end{covsubexamples}



Many original phoneme sequences are longer than 10 segments. It is possible to increase the limit of the model after how many segments to cut off segments. However, 30 segments is already quite long. There is no point in setting this limit too high as neural models like the CMU model still experience difficulties in processing sequences that are too long. Not only the sequence length is influenced but also the vocabulary size. Each flag will be added to the vocabulary which increases the vocabulary by a lot. This makes it harder for the model to pick one character from the vocabulary as output as there are more options available.  

It is therefore not surprising that the model did not perform well. In fact, the \ac{wer} of the model trained on the German data with feature version 1 was about double the \ac{wer} of the long model trained on the data with no features. Given this results, I decided not to train any more models for this feature version and look for another way of how to encode the features.

\subsection*{Version 2}
For the second attempt at encoding phonetic features I tried a different approach. Instead of adding two flags, I added only one flag after each phoneme. Also, this time the idea was not to add a unique flag to each phoneme but to encode only some features for all phonemes. This means that for some phonemes that have overlapping features the same flag was added. As the phonemes were not replaced by the flags, but the phonemes were kept in the text, there is no information lost. 

The intuition behind this second approach is to encode high-level patterns in phoneme sequences. For example, as vowels typically have some overlapping features, very similar vowels will be encoded using the same flag. Again, in order to enrich my data with the features, I followed specific steps which I will describe in more detail below: 

\begin{description}
    \item[\textsc{1. step}] As a very first step, I collected a list of all unique phonemes in PHOIBLE together with their features. I will later use this set to encode each phoneme.
    \item[\textsc{2. step}] In contrast to the first feature version, for the second feature version I computed the features based on the phoneme inventory of each language. Consequently, I collected additional sets all PHOIBLE phonemes for \textit{each language separately}. All of the following steps are conducted for each language separately.
    \item[\textsc{3. step}] As a next step, I collected a set of all phonemes for each language that are found in the training data for the models which is the WikiPron data in my case. 
    After this step, I have two sets of phonemes for each language:
    \begin{enumerate}
    \item one set of all phonemes in PHOIBLE for the respective language
        \item one set of all phonemes in the WikiPron training data for the respective language
    \end{enumerate}
    Taking the intersection of these two sets gives me the set of phonemes which are found in PHOIBLE and in the WikiPron data for that language.
    \item[\textsc{4. step}] For each phoneme in the WikiPron data, I got the PHOIBLE feature vector for that phoneme. If a phoneme is not found in PHOIBLE but in the WikiPron data, I just ignore that phoneme for the next steps.
    \item[\textsc{5. step}] This step is the key step of the current feature version. I decided what features to encode for each language performing this step. In table \ref{tab:enc_features}, I list all features that I encoded for each language. In order to choose appropriate features for each language, I calculated the Pearson correlation between the features of all phonemes for each language. The idea behind calculating the correlation between the features is to exclude features for one language that are not important for that language. If I find two features for the phonemes of a language that have a strong correlation, it means that those two features behave in a similar way across phonemes. For example:
    
    \begin{itemize}
        \item There is a phoneme /\textipa{N}/ that has feature `sonorant' marked as `+' and feature `continuant' as `-'.
        \item Then there is another phoneme /\textipa{n}/ that also has feature `sonorant' marked as `+' and feature `continuant' as `-'.
    \end{itemize}
    Based on the above very small example, we could argue that whenever the feature `sonorant' is `+', then the feature `continuant' is `-'. If we observe this pattern for those two features for many phonemes of a language, then there is a correlation between the two features `sonorant' and `continuant'. It means that when we only observe the feature `sonorant' to be `+', we can infer that the  feature `continuant' is `-'. For all features of all phonemes of a language, the calculation is a bit more complex but the intuition is the same. When we study table \ref{tab:enc_features} closely, we can actually see that for some languages (for example Chinese or German) only `sonorant' was encoded but not `continuant'. This means, that there actually is a correlation between `continuant' and at least one other feature in those languages which is why `continuant' was not encoded. Having two strongly correlating features in a dataset means to encode the same information twice. And this is exactly what we would like to avoid.
    
    I only encoded those features for a language whose correlation with no other feature within that language is higher than a certain threshold. For this experiment I set the correlation threshold to be 0.5. If the threshold was too high, for many languages, no correlating features were found. This is not surprising as features in general do not make a lot of sense if they are all encoding the same information. This is why I set the threshold relatively low which means I did not encode a lot of features. 
    
    The reason why I suggest that this feature encoding strategy can be used to find important features for a language is that all languages only use a relatively small subset of phonemes. This means that many features are not important for a language as those features are not used to distinguish meaningful sounds in that language. In order to recognize high-level phonetic patterns in a language, only the most distinctive features are important. 
    
    \item[\textsc{6. step}] Once I have all the features for one language that I want to encode, I can actually encode them. This step is similar to the encoding of the features of the first version. I get all combinations of the set of features to encode for each language (similar as in table \ref{tab:feature-encoding}) and assign it a different capitalized two-letter string. Table \ref{tab:f2-strings} lists a few example strings for how the input data changed after enriching it with features.  
\end{description}

There are still a few phonemes that are not in the PHOIBLE set that are in the WikiPron data. These will just not be encoded. An exception is the length marker. There are a few long vowels that are not found in the PHOIBLE data. As there is the `long' feature in PHOIBLE, it is easily possible to use the base character feature vector and mark the `long' as `+'. This is the only thing I changed for the following experiments. All other features vectors are just used as they are retrieved from PHOIBLE. I encoded the features in a jupyter notebook which is in my GitHub repository\myfootnote{\url{https://github.com/theDebbister/masterThesis/blob/master/data/MA_Encode_features.ipynb}}.

\tab{tab:f2-strings}{This table shows parallel examples for the first few and the last few entries of the German broad WikiPron data file for the feature encoding version 2. It shows the graphemes, the unchanged phoneme sequence and the phoneme sequence with features. In the original file, the phoneme column is omitted. The feature encodings are not unique for each phoneme. I marked an example for the same encoding for two phonemes in red. Note that the two phonemes are very similar (see figure \ref{fig:ipa_chart}), thus have similar features and consequently the same feature encoding.}{
\begin{tabular}{|lll|}
\hline
\textbf{Grapheme}   & \textbf{Phoneme}           & \textbf{Phonemes with features}                    \\\hline\hline
a          & \textipa{P a:}                 & \textipa{P } AK \textipa{a: } AM                                \\
aa         & \textipa{a:}                   & \textipa{a: } AM                                     \\
aachen     & \textipa{a: x @ n}             & \textipa{a: } AM \textipa{x } AJ \textipa{@ } AL \textipa{n } AG                      \\
aachener   & \textipa{a: x @ n 5}           & \textipa{a: } AM \textipa{x } AJ \textipa{@ } AL \textipa{n } AG \textipa{5} AN          \\
aachenerin & \textipa{a: x @ n @ K I n}     & \textipa{a: } AM \textipa{x } AJ \textipa{@ } AL \textipa{n } AG \textipa{@ } AL \textipa{K } AJ \textipa{I } AL \textipa{n } AG  \\
...        & ...                            & ...                                       \\
übungen    & \textipa{P y: b U N @ n}       & \textipa{P } AK \textipa{y: } BA \textipa{b } AH \textipa{U } AN {\color{red}\textipa{N } AG} \textipa{@ } AL {\color{red}\textipa{n } AG}       \\
übungsbuch & \textipa{y: b U N s b u: x}    & \textipa{y:} BA \textipa{b } AH \textipa{U } AN \textipa{N } AG \textipa{s } AJ \textipa{b } AH \textipa{u: } BA \textipa{x } AJ  \\
üppig      & \textipa{Y p I \|)c}           & \textipa{Y } AN \textipa{p } AH \textipa{I } AL  \textipa{\|)c}                        \\
üsselig    & \textipa{Y z @ l I \|)c}       & \textipa{Y } AN \textipa{z } AJ \textipa{@ } AL \textipa{l } AE \textipa{I }  AL \textipa{\|)c }    \\
œuvre      & \textipa{{\oe}: v K @}    & \textipa{{\oe}:} BA \textipa{v } BC \textipa{K } AJ \textipa{@ } AL \\\hline              
\end{tabular}
}{Example data feature version 2}


While encoding these features, I noticed that some phonemes are not listed as individual phonemes in PHOIBLE but are listed as allophones of one phoneme that is in PHOIBLE. This means that some broad transcriptions contain allophones which, ideally, should not be the case (see section \ref{sec:ipa}). Also, it means that I did not have features for these phonemes which means that those were not encoded. It would be possible to replace all allophones for each language with the respective phoneme. However, this would have gone beyond the scope of this present thesis.

\tab{tab:enc_features}{The table shows all features that were encoded for each language for feature version 2. Some of them are encoded for all languages, those are at the top of the table. All features that are listed for each language have a correlation lower than 0.5 with all other features listed for that language.}{
\vspace{-0.7cm}
\hspace*{-2cm}
\begin{tabularx}{1.3\textwidth}{|llX|}
\hline
\textbf{ISO 396-3} & \textbf{Type} & \textbf{Features}                                                                                                                  \\
\hline
\hline
all langs     & -             & tone, stress, syllabic, short, consonantal, tap, trill, lateral, labial, epilaryngealSource, spreadGlottis, loweredLarynxImplosive \\
\hline
\hline
cmn                & broad         & long, continuant, nasal, constrictedGlottis, raisedLarynxEjective                                                                  \\
deu                & narrow        & long, continuant, nasal, constrictedGlottis, raisedLarynxEjective                                                                  \\
ell                & broad         & long, sonorant, continuant, delayedRelease, nasal, coronal, dorsal, constrictedGlottis, raisedLarynxEjective                       \\
eng uk             & broad         & long, continuant, nasal, constrictedGlottis, raisedLarynxEjective                                                                  \\
eng uk             & narrow        & long, continuant, nasal, constrictedGlottis                                                                                        \\
eng us             & broad         & long, continuant, nasal, constrictedGlottis, raisedLarynxEjective                                                                  \\
eng us             & narrow        & long, continuant, nasal, constrictedGlottis                                                                                        \\
eus                & broad         & long, continuant, delayedRelease, nasal, constrictedGlottis, raisedLarynxEjective                                                  \\
fin                & broad         & long, continuant, nasal, constrictedGlottis, raisedLarynxEjective                                                                  \\
fin                & narrow        & long, continuant, delayedRelease, nasal, constrictedGlottis, raisedLarynxEjective                                                  \\
fra                & broad         & long, nasal, constrictedGlottis, raisedLarynxEjective                                                                              \\
hin                & broad         & constrictedGlottis, raisedLarynxEjective                                                                                           \\
hin                & narrow        & nasal, constrictedGlottis, raisedLarynxEjective                                                                                    \\
ind                & broad         & long, continuant, nasal, constrictedGlottis, raisedLarynxEjective                                                                  \\
ind                & narrow        & long, continuant, nasal, constrictedGlottis, raisedLarynxEjective                                                                  \\
jpn                & narrow        & long, delayedRelease, coronal, dorsal, constrictedGlottis, raisedLarynxEjective                                                    \\
kat                & broad    & long, continuant, delayedRelease, nasal, coronal, constrictedGlottis                                                               \\
kor                & narrow        & continuant, nasal, constrictedGlottis, raisedLarynxEjective                                                                        \\
mya                & broad    & long, delayedRelease, nasal, raisedLarynxEjective                                                                                  \\
rus                & narrow        & long, continuant, nasal, coronal, dorsal, constrictedGlottis, raisedLarynxEjective                                                 \\
spa ca             & broad         & long, sonorant, continuant, delayedRelease, nasal, dorsal, constrictedGlottis, raisedLarynxEjective                                \\
spa ca             & narrow        & long, sonorant, continuant, nasal, constrictedGlottis, raisedLarynxEjective                                                        \\
spa la             & broad         & long, sonorant, continuant, nasal, coronal, dorsal, constrictedGlottis, raisedLarynxEjective                                       \\
spa la             & narrow        & long, sonorant, continuant, nasal, constrictedGlottis, raisedLarynxEjective                                                        \\
tgl                & broad         & long, continuant, delayedRelease, nasal, constrictedGlottis, raisedLarynxEjective                                                  \\
tgl                & narrow        & long, sonorant, continuant, nasal, dorsal, constrictedGlottis, raisedLarynxEjective                                                \\
tha                & broad         & nasal, constrictedGlottis, raisedLarynxEjective                                                                                    \\
tur                & broad         & long, continuant, nasal, constrictedGlottis, raisedLarynxEjective                                                                  \\
tur                & narrow        & long, nasal, constrictedGlottis, raisedLarynxEjective                                                                              \\
vie                & narrow        & long, nasal, coronal, constrictedGlottis, raisedLarynxEjective                                                                     \\
zul                & broad         & continuant, delayedRelease, nasal, coronal, constrictedGlottis, raisedLarynxEjective  \\
\hline                                            
\end{tabularx}
\vspace{-0.2cm}
}{Encoded features per language}


\section{Results}

For the evaluation of my models, I first used the WikiPron test sets. The test set is a subset of the entire WikiPron dataset for one language as explained in section \ref{sec:train-eval}. In addition, I tested each model on the respective \ac{nws} story if available in that language. Below, I first discuss the results separately for the short (setting 1, setting 3 and setting 7) and the long (setting 2, setting 4 and setting 8) models. I will mostly compare the \ac{wer} scores of the model, as this is common in research. Also, the \ac{per} scores are a bit harder to interpret, as the edit distance it involves is not very intuitively understandable as there are many different ways of how to get to the same result. 

To evaluate the models that were trained on data containing feature flags, I cleaned out the features to calculate the scores on the predicted phonemes only. Leaving in the feature flags would make it difficult to compare it to the other models' results.
\subsection{Results: short models}
Table \ref{tab:res_short_full} shows the results for all models that I trained for 10,000 steps. It is interesting that the results for all three data types (uncleaned, cleaned and feature data) are actually very similar. I was surprised by the fact that for a vast majority of the languages the version trained on the cleaned data did not improve the performance although the cleaning was supposed to make the data more consistent. For example, excluding duplicate words with different phoneme sequences should remove some ambiguities. However, it is also possible that it actually introduced more ambiguities as I randomly took one version of the duplicates which might be an uncommon one.  

An interesting result is the one for Zulu [zul]. The cleaned model performed a lot better than the other two. As this result seemed rather strange, I had a close look at the data again. Manually comparing the test prediction for Zulu to the reference phonemes (there are only 167 words in the test set) reveals that the model almost exclusively got the tones wrong. Those were excluded in the cleaned version which might explain the wide gap. Finnish [fin] had a very similar result but the gap was not as large. This finding shows how important appropriate preprocessing is.

\tab{tab:res_short_full}{This table shows the results for setting 1 (BS), setting 3 (BS-clean) and setting 7 (F2). These are the short models.}{
\hspace*{-1.1cm}
\begin{tabularx}{1.1\textwidth}{|l|X||r|r||X|X||r|r|}
\hline
\textbf{Iso 639-3} & \textbf{Type WikiPron} & \textbf{WER BS}       & \textbf{PER BS} & \textbf{WER BS-clean} & \textbf{PER BS-clean} & \textbf{WER F2} & \textbf{PER F2} \\ \hline \hline
cmn    & broad  & 19.6 & 4.5  & \textbf{18.1} & 5.0  & 20.3 & 6.0  \\
deu    & broad  & 40.3 & 5.4  & \textbf{38.1} & 5.3  & 40.2 & 6.6  \\
deu    & narrow & \textbf{52.1} & 7.5  & 59.9 & 9.4  & 59.9 & 10.4 \\
ell    & broad  & \textbf{9.8}  & 0.9  & 10.4 & 1.0  & 11.5 & 1.1  \\
eng us & broad  & 54.4 & 10.6 & \textbf{53.0} & 10.8 & 57.7 & 12.1 \\
eng us & narrow & 84.6 & 32.0 & \textbf{84.2} & 31.8 & 86.6 & 32.2 \\
eng uk & broad  & \textbf{48.6} & 9.5  & 50.0 & 10.1 & 51.8 & 11.1 \\
eng uk & narrow & \textbf{88.8} & 32.2 & 93.8 & 34.5 & 90.5 & 31.2 \\
eus    & broad  & \textbf{19.4} & 2.8  & 22.8 & 2.8  & 23.6 & 4.0  \\
fin    & broad  & \textbf{5.8}  & 0.4  & 11.3 & 1.3  & 11.5 & 3.1  \\
fin    & narrow & 14.3 & 1.0  & \textbf{7.1}  & 0.7  & 17.1 & 4.3  \\
fra    & broad  & \textbf{7.2}  & 1.0  & 8.8  & 1.3  & 9.1  & 1.6  \\
hin    & narrow & 8.4  & 1.4  & 10.1 & 1.6  & \textbf{8.3}  & 1.3  \\
hin    & broad  & \textbf{5.6} & 1.2  & 7.3  & 1.2  & 6.9  & 1.2  \\
ind    & broad  & \textbf{35.3} & 5.3  & 36.3 & 4.8  & 38.6 & 5.7  \\
ind    & narrow & \textbf{43.5} & 5.5  & 44.1 & 6.0  & 44.3 & 6.1  \\
jpn    & narrow & \textbf{6.6}  & 0.9  & 6.7  & 0.9  & 6.8  & 1.1  \\
kat    & broad  & \textbf{0.3}  & 0.2  & 1.1  & 0.2  & 1.7  & 0.9  \\
kor    & narrow & 28.7 & 4.6  & 27.0 & 4.4  & \textbf{26.7} & 4.9  \\
mya    & broad  & 34.2 & 7.4  & \textbf{34.1} & 7.1  & 87.5 & 18.8 \\
rus    & narrow & \textbf{14.8} & 1.7  & 16.4 & 1.9  & 22.0 & 3.6  \\
spa ca & broad  & \textbf{2.3}  & 0.3  & 2.7  & 0.4  & 5.3  & 1.5  \\
spa ca & narrow & \textbf{3.6}  & 0.5  & 9.1  & 1.5  & 8.4  & 1.6  \\
spa la & broad  & \textbf{2.6}  & 0.3  & 2.8  & 0.5  & 4.4  & 1.0  \\
spa la & narrow & 3.3  & 0.4  & \textbf{3.2}  & 0.3  & 5.4  & 0.8  \\
tgl    & broad  & \textbf{33.3} & 5.9  & 33.4 & 5.1  & 34.8 & 5.6  \\
tgl    & narrow & \textbf{42.9} & 6.6  & 51.1 & 7.6  & 50.3 & 7.2  \\
tha    & broad  & 14.1 & 2.9  & \textbf{13.0} & 2.8  & 15.3 & 3.7  \\
tur    & broad  & \textbf{52.8} & 8.0  & 53.4 & 7.9  & 54.3 & 8.6  \\
tur    & narrow & 57.9 & 8.4  & 53.9 & 7.9  & \textbf{52.0} & 8.5  \\
vie    & narrow & \textbf{3.0}  & 1.3  & \textbf{3.0}  & 1.5  & 4.3  & 2.0  \\
zul    & broad  & 61.7 & 11.5 & \textbf{10.4} & 1.1  & 95.1 & 13.7 \\ \hline
\end{tabularx}
}{Results: short models}

Table \ref{tab:baseline_short} shows a comparison between my short models and the \ac{sigm} models. The results for the shared task are the same no matter to which of my models I compare them. As we will see in the next section, my long models performed better than the short ones presented in the current section. Therefore, I will not discuss the results at this point for the short models compared with the shared task. I will only discuss my results of the long models compared with the \ac{sigm} models.

\tab{tab:baseline_short}{The table shows the \ac{wer} results for the best of my short models compared with the \ac{sigm} 2020 and 2021 results. For each language the best score is reported no matter what year or what model from the \ac{sigm} task. My models and the \ac{sigm} models were trained on WikiPron data. However, the \ac{sigm} data was preprocessed in a slightly different way which means that the results are not directly comparable. All references for the \ac{sigm} models can be found in tables \ref{tab:sota} and \ref{tab:sota_table_long}.}{
\begin{tabular}{|rrrrl|}
\hline
\textbf{ISO396-3} & \textbf{BS WER} &  \textbf{SIG WER} & \textbf{Transcription type} & \textbf{Model name SIG} \\
\hline
\hline
eng (us) & 53.00 & \textbf{37.43} & broad & DP21 \\
fra & 7.20 &  \textbf{5.11} & broad & DeepSPIN20 \\
ell & \textbf{9.80} &  18.67 & broad & IMS20 \\
kat & 0.30 &  \textbf{0.00} & broad & BS21/CL21\\
hin & 5.60 &  \textbf{5.11} & broad & IMS20\\
jpn & 6.60 &  \textbf{4.89} & narrow & DeepSPIN20\\
kor & 26.70 &  \textbf{16.20} & narrow & CL21\\
vie & 3.00 &  \textbf{0.89} & narrow & DeepSPIN20\\
\hline
\end{tabular}
}{Short models comparison}

\subsection{Results: long models}

Table \ref{tab:res_long_full} shows the results for all models that I trained for 200,000 steps. The results and the differences between the models are very similar to the short models. The observation for Zulu [zul] is still the same. In fact, the long Zulu model trained on the uncleaned data performed even worse than its short equivalent.

The largest improvement from the short to the long models was for Russian [rus]. Russian had by far the largest dataset with more than 400,000 samples. It makes sense that for this large dataset, the model takes more time to extract all the relevant information. But it also shows that longer training times are not always equally helpful.

\tab{tab:res_long_full}{This table shows the results for setting 2 (BS), setting 4 (BS-clean) and setting 8 (F2). These are the long models}{
\hspace*{-1.5cm}
\begin{tabularx}{1.2\textwidth}{|l|X||r|r||X|X||r|r|}
\hline
\textbf{ISO 639-3} & \textbf{Type WikiPron} & \textbf{WER BS}       & \textbf{PER BS} & \textbf{WER BS-clean} & \textbf{PER BS-clean} & \textbf{WER F2} & \textbf{PER F2} \\ \hline \hline
cmn       & broad         & 17.6         & \textit{3.9}    & \textbf{17.0}         & \textit{4.0}          & 18.1   & \textit{4.4}    \\
deu       & broad         & \textbf{37.1}& \textit{4.8}    & 38.1         & \textit{5.0}          & 38.6   & \textit{5.9}    \\
deu       & narrow        & \textbf{52.2}& \textit{7.1}    & 56.9         & \textit{8.4}          & 55.9   & \textit{9.4}    \\
ell       & broad         & \textbf{7.1} & \textit{0.6}    & 9.0          & \textit{0.7}          & 9.1    & \textit{0.8}    \\
eng us    & broad         & \textbf{50.7}& \textit{9.5}    & 51.2         & \textit{10.2}         & 51.3   & \textit{10.4}   \\
eng us    & narrow        & 84.6         & \textit{31.4}   & \textbf{84.2}& \textit{33.1}         & 84.9   & \textit{32.3}   \\
eng uk    & broad         & \textbf{45.5}& \textit{8.6}    & 47.5         & \textit{9.2}          & 47.6   & \textit{9.7}    \\
eng uk    & narrow        & \textbf{90.3}& \textit{30.2}   & 94.8         & \textit{35.3}         & 93.7   & \textit{34.2}   \\
eus       & broad         & 21.2         & \textit{2.7}    & \textbf{19.6}& \textit{2.3}          & 21.7   & \textit{3.2}    \\
fin       & broad         & \textbf{2.8} & \textit{0.2}    & 3.3          & \textit{0.3}          & 8.9    & \textit{3.1}    \\
fin       & narrow        & \textbf{3.2} & \textit{0.3}    & 3.9          & \textit{0.4}          & 9.0    & \textit{3.5}    \\
fra       & broad         & \textbf{5.3} & \textit{0.7}    & 5.8          & \textit{0.8}          & 5.9    & \textit{0.8}    \\
hin       & narrow        & \textbf{7.7} & \textit{1.2}    & 8.4          & \textit{1.4}          & 8.4    & \textit{1.3}    \\
hin       & broad         & \textbf{4.4} & \textit{0.7}    & 6.4          & \textit{1.0}          & 6.3    & \textit{1.0}    \\
ind       & broad         & 37.9         & \textit{5.3}    & \textbf{34.9}& \textit{5.3}          & 39.3   & \textit{6.1}    \\
ind       & narrow        & 43.1         & \textit{5.4}    & \textbf{43.0}& \textit{5.6}          & 43.1   & \textit{5.6}    \\
jpn       & narrow        & \textbf{6.5} & \textit{0.6}    & 6.8          & \textit{0.6}          & 6.6    & \textit{0.8}    \\
kat       & broad         & \textbf{0.0} & \textit{0.0}    & \textbf{0.0} & \textit{0.0}          & 1.0    & \textit{0.8}    \\
kor       & narrow        & \textbf{23.4}& \textit{4.1}    & 25.3         & \textit{4.4}          & 25.8   & \textit{4.6}    \\
mya       & broad         & \textbf{35.1}& \textit{6.5}    & 36.0         & \textit{6.9}          & 88.0   & \textit{17.4}   \\
rus       & narrow        & \textbf{1.9} & \textit{0.2}    & 2.4          & \textit{0.3}          & 5.0    & \textit{1.5}    \\
spa ca    & broad         & \textbf{1.1} & \textit{0.1}    & 1.3          & \textit{0.1}          & 2.2    & \textit{0.7}    \\
spa ca    & narrow        & 2.3          & \textit{0.3}    & \textbf{2.2}          & \textit{0.3}          & 2.8    & \textit{0.6}    \\
spa la    & broad         & \textbf{1.4} & \textit{0.1}    & 1.5          & \textit{0.2}          & 1.9    & \textit{0.6}    \\
spa la    & narrow        & \textbf{2.6} & \textit{0.3}    & 2.7          & \textit{0.3}          & 2.7    & \textit{0.4}    \\
tgl       & broad         & \textbf{28.4}& \textit{4.6}    & 31.2         & \textit{5.2}          & 33.9   & \textit{5.0}    \\
tgl       & narrow        & \textbf{45.5}& \textit{6.4}    & 47.3         & \textit{7.0}          & 48.6   & \textit{6.9}    \\
tha       & broad         & 12.5         & \textit{2.6}    & \textbf{11.1}         & \textit{2.5}          & 12.1   & \textit{3.1}    \\
tur       & broad         & 50.6         & \textit{7.8}    & 52.3         & \textit{7.5}          & \textbf{49.1}   & \textit{7.2}    \\
tur       & narrow        & 55.1         & \textit{7.6}    & 55.6         & \textit{8.3}          & \textbf{54.8}   & \textit{8.0}    \\
vie       & narrow        & \textbf{1.5} & \textit{0.8}    & 1.6          & \textit{0.8}          & 2.6    & \textit{1.5}    \\
zul       & broad         & 65.9         & \textit{10.7}   & \textbf{9.8}          & \textit{0.9}          & 91.4   & \textit{12.0}   \\ \hline
\end{tabularx}
}{Results: long models}

Table \ref{tab:baseline_long_sig} shows a comparison of my best long model compared with the best SIGMORPHON shared task result for different languages. For three of eight languages in table \ref{tab:baseline_long_sig}, my model reached a better or the same score. The \ac{sigm} Korean [kor] model performed quite a lot better than mine. This result can be explained as the \ac{sigm} model was not trained on the Korean logograms but those were converted to hangul characters which is a Latin alphabet representation of Korean. As explained in section \ref{section:sig}, converting logographic scripts to alphabets can lead to better results as it reduces the vocabulary size of the model. My result for English is a lot worse as well. Generally, English (as well as German) seems to be a language that is relatively difficult to pronounce as the results are often a lot worse than those of other languages. In the present case, the difference might result from the fact that the \ac{sigm} model for English was an ensemble and it was only trained for English and therefore optimized to work for the English language. 

My model performs better on Modern Greek than the \ac{sigm} model. The \ac{sigm} model was trained on only 800 samples. I had more than 10,000 samples. This result aligns with cutting edge research, as \citet{Ashby-Bartley.2021} have found that 800 samples are not enough to train a decent \ac{g2p} model. My results confirm this finding as my model trained on 10,000 samples of the same data was \textit{considerably} better than the model trained on only 800 samples. The only difference between the training sets of the models was that I applied different preprocessing (see section \ref{preprocess}). 

\tab{tab:baseline_long_sig}{The table shows the \ac{wer} results for the best of my long models compared with the \ac{sigm} 2020 and 2021 results. For each language the best score is reported no matter what year or what model from the \ac{sigm} task. My models and the \ac{sigm} models were trained on WikiPron data. However, the \ac{sigm} data was preprocessed in a slightly different way which means that the results are not directly comparable. All references for the \ac{sigm} models can be found in tables \ref{tab:sota} and \ref{tab:sota_table_long}.}{
\begin{tabular}{|rrrrl|}
\hline
\textbf{ISO396-3} & \textbf{BS WER} & \textbf{SIG WER} & \textbf{Transcription type} & \textbf{Model name SIG} \\
\hline
\hline
eng (us) & 50.70 &  \textbf{37.43} & broad & DP21\\
fra & 5.30 &  \textbf{5.11} & broad & DeepSPIN20\\
ell & \textbf{7.10} &  18.67 & broad & IMS20\\
kat & \textbf{0.00} &  \textbf{0.00} & broad & BS21/CL21 \\
hin & \textbf{4.40} &  5.11 & broad & IMS2\\
jpn & 6.50 &  \textbf{4.89} & narrow & DeepSPIN2\\
kor & 23.40 & \textbf{16.20} & narrow & CL21\\
vie & 1.50 &  \textbf{0.89} & narrow & DeepSPIN20\\
\hline
\end{tabular}
}{Long models comparison}

\subsection{Results: NWS tests}
Table \ref{tab:nws_test} presents the results for the tests on the \ac{nws} stories. The models do not perform very well on these stories in general. Concerning the data types, there is no real pattern in how the models trained on each of the data types perform. The clean models seem to perform a bit better than the other two types. Also, as the texts are really short, the performance of all models is often very similar.

Still, when compared to the coverage experiment in section \ref{sec:coverage}, many \ac{g2p} models perform quite a lot better than if we just used the WikiPron data as a look-up table. Although this is not surprising, it is good that we can confirm that those models are able to extract a lot of information that is not found in a simple look-up table. What is surprsing is that there are a few languages link broad Hindi [hin] or Vietnamese [vie] where the \ac{wer} of the \ac{g2p} models is higher or very similar to the coverage experiment. Neither of these models performed particularly bad when tested on the WikiPron test set (\ac{wer} long model Hindi: 4.4, Vietnamese: 1.5), in fact the results were very good. In addition, both training sets contain more than 10,000 samples. The most plausible explanation is that the transcription conventions used in the WikiPron data and the \ac{nws} stories are just very different.  

\begin{center}
\tab{tab:nws_test}{In this table I present the results for all models when they predicted the NWS stories. The models are explained in section \ref{sec:train-settings} where I explain the different training settings. `Long' corresponds to setting 2, `short' to setting 1, `Long clean' to setting 4, `Short clean' to setting 3, `F2 long' to setting 8 and `F2 short' to setting 7.}{
\hspace*{-2cm}
\begin{tabularx}{1.3\textwidth}{|Xl||ll||ll||ll||ll||ll||ll|}
\hline
\textbf{ISO 396-3} & \textbf{Type} & \multicolumn{2}{c||}{\textbf{Long}} & \multicolumn{2}{c||}{\textbf{Short}} & \multicolumn{2}{c||}{\textbf{Long clean}} & \multicolumn{2}{c||}{\textbf{Short clean}} & \multicolumn{2}{c||}{\textbf{F2 long}} & \multicolumn{2}{c|}{\textbf{F2 short}} \\ \hline
                   &               & \textbf{WER}       & \textbf{PER}     & \textbf{WER}       & \textbf{PER}      & \textbf{WER}        & \textbf{PER}      & \textbf{WER}        & \textbf{PER}       & \textbf{WER}      & \textbf{PER}     & \textbf{WER}       & \textbf{PER}     \\ \hline \hline
cmn                & broad         & 95.0               & 72.5             & 100                & 73.8              & 94.1                & 45.2              & 86.1                & 40.5               & 93.1              & 39.2             & \textbf{84.2}      & 36.4             \\
deu                & broad         & 74.1               & 26.0             & 76.9               & 28.0              & 72.2                & 24.5              & \textbf{63.0}       & 23.4               & 72.0              & 20.8             & 65.4               & 19.8             \\
deu                & narrow        & \textbf{50.9}      & 16.2             & 54.6               & 16.4              & 63.9                & 19.6              & 63.9                & 20                 & 57.0              & 16.6             & 59.8               & 17.9             \\
ell                & broad         & 78.9               & 32.8             & \textbf{75.4}      & 31.1              & \textbf{75.4}       & 29.2              & \textbf{75.4}       & 29.5               & 77.0              & 17.3             & 77.0               & 18.3             \\
eng us             & broad         & \textbf{66.4}      & 22.3             & 87.6               & 32.5              & 84.1                & 27                & 87.6                & 30.4               & 83.9              & 20.9             & 92.0               & 24.7             \\
eng us             & narrow        & \textbf{94.7}      & 47.5             & 97.3               & 44.5              & 97.3                & 51.8              & 97.3                & 47.9               & 97.3              & 45.1             & 97.3               & 44.1             \\
eus                & broad         & 36.8               & 4.2              & 48.3               & 5.9               & 36.8                & 4.8               & \textbf{34.5}       & 4.7                & 34.9              & 5.1              & 34.9               & 4.8              \\
fra                & broad         & 37.0               & 15.6             & 45.4               & 17.7              & \textbf{34.3}       & 14.3              & 48.1                & 19.3               & 34.6              & 11.6             & 48.6               & 15.5             \\
hin                & narrow        & 91.1               & 62.6             & 91.1               & 62.6              & 90.3                & 58.5              & 90.3                & 58.5               & \textbf{90.2}     & 29.9             & \textbf{90.2}      & 30.4             \\
hin                & broad         & 96.0               & 56.1             & 96.0               & 55.8              & 95.2                & 57.8              & 95.2                & 57.8               & \textbf{95.1}     & 47.0             & \textbf{95.1}      & 47.0             \\
ind                & broad         & \textbf{81.5}      & 30.8             & \textbf{81.5}      & 27.9              & \textbf{81.5}       & 29.7              & \textbf{81.5}       & 30.1               & 82.2              & 17.8             & 82.2               & 19.4             \\
ind                & narrow        & 85.2               & 33.0             & \textbf{82.4}      & 32.2              & \textbf{82.4}       & 31.3              & \textbf{82.4}       & 31.0               & 86.0              & 23.8             & 86.0               & 23.6             \\
kat                & broad         & 64.2               & 14.6             & 64.2               & 14.6              & \textbf{58.2}       & 11.5              & \textbf{58.2}       & 11.5               & 59.1              & 7.5              & 59.1               & 7.8              \\
kor                & narrow        & 100.0              & 52.5             & 100.0              & 52.7              & 100.0               & 52.5              & 100.0               & 51.4               & 100.0             & 100.0            & 100.0              & 100.0            \\
mya                & broad         & 97.5               & 34.4             & 97.5               & 36.5              & \textbf{95.0}       & 21.3              & \textbf{95.0}       & 23.2               & 97.4              & 29.4             & 97.4               & 35.8             \\
rus                & narrow        & 91.6               & 37.2             & 91.6               & 37.3              & \textbf{89.5}       & 36.3              & 91.6                & 37.6               & 91.5              & 35.7             & 91.5               & 37.1             \\
spa ca             & broad         & \textbf{27.8}      & 5.1              & 28.9               & 5.4               & \textbf{27.8}       & 5.1               & \textbf{27.8}       & 5.1                & 32.3              & 5.4              & 32.3               & 5.4              \\
spa ca             & narrow        & 56.7               & 17.3             & \textbf{53.6}      & 16.5              & 58.8                & 16.2              & 54.6                & 16.2               & 65.6              & 17.3             & 54.2               & 16.1             \\
tha                & broad         & 99.4               & 62.3             & 99.4               & 62.2              & 56.9                & 19.0              & 56.9                & 19.1               & 57.2              & 16.3             & \textbf{56.6}      & 16.4             \\
tur                & broad         & 53.8               & 7.8              & 63.1               & 10.1              & \textbf{46.2}       & 6.7               & 55.4                & 8.7                & 56.2              & 9.1              & 64.1               & 9.2              \\
tur                & narrow        & 76.9               & 15.6             & \textbf{66.2}      & 11.6              & 76.9                & 11.9              & 75.4                & 13.7               & 75.0              & 11.9             & 68.8               & 12.1             \\
vie                & narrow        & 100.0              & 99.8             & 100.0              & 99.8              & \textbf{79.5}       & 44.2              & \textbf{79.5}       & 44.2               & 80.2              & 36.1             & 80.2               & 36.1  \\
\hline   

\end{tabularx}
}{NWS test results} 

\end{center}  


\subsection{Results: overall}
All models trained on one data type show a similar performance no matter how long they were trained. The most notable observation is that the difference between the performance of the long and the short models is rather small. For some languages, the best short model performed even better than the long one or the results were the same. This was the case for narrow German [deu], narrow US English [eng us], narrow UK English [eng uk], broad Goizueta Basque [eus], broad Burmese [mya], narrow Tagalog [tgl] and narrow Turkish [tur]. I find it surprising that most of these languages are the narrow variants as those are more detailed. I would expect a model trained for longer to catch more details. However, none of the scores were really low, the lowest being 19.4 for Basque. Additionally, the differences between the short and long models for these languages are never larger than 2.8 percentage points.

The feature models did not show any relevant improvement compared to the models trained without features. They did not worsen the results but performed very similarly. When the long feature models are compared to the short ones, the models trained for longer reached a better performance for almost all languages. As the input data is more complex because I added the features, it might be worth exploring even longer training times. However, also the long trained models without features reached a better performance than the short models without features. This means that it could be possible that the feature models would still not outperform the models trained without features if the models for both data types were trained for longer. 






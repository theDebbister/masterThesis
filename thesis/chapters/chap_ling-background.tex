\newchap{Linguistic Background}
\label{chap:ling-background}

\section{The relation of spoken and written language}
\review{corpus linguistics and quantitative analysis. Remember that writing systems came only much later compared to language in general. Can they capture language as such well enough? Computational linguistics deals mostly with written languages, what does linguistics say and do?}

Whenever we study language we look at samples of that language. It is simply impossible to study an \textit{entire} language as we would need all texts that were ever produced in that language. Consequently, we need to ask ourselves how much material of a language is enough to study it properly \citep{baird_evans_greenhill_2021}. In addition, language material can represent written or spoken language. We will see later on in this chapter that mapping a spoken language to its written representation is far from easy and never perfect. \citet{baird_evans_greenhill_2021} focus on answering the question how much phonetic data is needed to represent a language well. 

\section{Phonemes and syllables}
\label{phonology}
Given that phonetics and phonology is a sub-area of traditional linguistics and often only touched on superficially in computational linguistics, I will summarise the most important assumptions and terms concerning said field. A very important terminological distinction is between phonetics and phonology. While phonetics refers to the study of actual sounds, phonology refers to the study of sound \textit{systems}. In phonetics, it is not so much important what the different sounds mean, but how they are produced and perceived and what different sounds a human being can produce and perceive at all. When it comes to human communication using spoken language, many of these sounds are not actually used to produce distinguishable meaning. This is why on the other hand phonology is important to describe the set of distinguishable sounds that make up a language. For example: the letter /r/ in English can be pronounced in many different ways. None of those pronunciations produces a change in meaning. This means that there exist many different \textit{phonetic} sounds but only one \textit{phonological} or \textit{phonemic}. Those sounds are referred to as phone and phoneme respectively. While there are infinitely many phones there are only finitely many phonemes in a language. Sounds that can be interchanged with another sound without changing the meaning are referred to as `allophones'. Not all different possible sounds are actually considered qualitatively `good' sounds of a language. Usually there is a subset of all possible phones that is accepted as `good quality sounds' within all different dialects of a language \citep{Intro.2007}. An obvious example being loudness: Although very silent speech produces correct phones, these are not `good quality' as they simply cannot be understood. Or speaking in English with hardly any mouth and tongue movement. Although this produces understandable sound, it is not generally considered good speech. 

It is important to note at this point that the terms phonetic and phonemic respectively phone and phoneme are sometimes used interchangeably. Their linguistic definition as given above is clear while the definition on the computational side is often less strict. Strictly speaking phonemic transcriptions are not allowed to contain allophones but should write the respective phoneme. This will not always be the case when it comes to data used in language technology \citep{Lee&Ashby.2020}. 


\begin{figure}[t!]
{\large
\vowelvunit=4em
\vowelhunit=4em
\begin{center}
\begin{vowel}
	\putcvowel[l]{i}{1}
\putcvowel[r]{y}{1}
\putcvowel[l]{e}{2}
\putcvowel[r]{\o}{2}
\putcvowel[l]{\textepsilon}{3}
\putcvowel[r]{\oe}{3}
\putcvowel[l]{a}{4}
\putcvowel[r]{\textscoelig}{4}
\putcvowel[l]{\textscripta}{5}
\putcvowel[r]{\textturnscripta}{5}
\putcvowel[l]{\textturnv}{6}
\putcvowel[r]{\textopeno}{6}
\putcvowel[l]{\textramshorns}{7}
\putcvowel[r]{o}{7}
\putcvowel[l]{\textturnm}{8}
\putcvowel[r]{u}{8}
\putcvowel[l]{\textbari}{9}
\putcvowel[r]{\textbaru}{9}
\putcvowel[l]{\textreve}{10}
\putcvowel[r]{\textbaro}{10}
\putcvowel{\textschwa}{11}
\putcvowel[l]{\textrevepsilon}{12}
\putcvowel[r]{\textcloserevepsilon}{12}
\putcvowel{\textsci\ \textscy}{13}
\putcvowel{\textupsilon}{14}
\putcvowel{\textturna}{15}
\putcvowel{\ae}{16}
\end{vowel}
\end{center}}
\stepcounter{myfigure}
\caption[Vowel chart]{The figure represents the vowel diagram as presented by the \ac{ipa}. The chart is meant to represent the vocal cavity of a human being from the side, with the mouth opening to the left. The edges and consequently the vowels are named analogous to the position of the tongue in the vocal cavity. The upper left vowel is called close front vowel. The upper right vowel is called close back vowel. The two vertical lines in the middle are half-close respectively half-open. The middle vertical line is simply called mid. \review{I am not sure if I should keep it, if I do, I will add a consonant chart as well. MAybe I'll put it into the appendix}}
\label{fig:vowel-diagram}
\end{figure}

\subparagraph{Vowels and consonants} Each phoneme can be described based on different categories. A well-known distinction is that between vowels and consonants. Both of these are again categorized differently. The schema for vowels and consonants is inspired by the human vocal cavity. The terms to describe vowels sounds are based on the position of the tongue in the mouth and if the lips are rounded or not. Using those two categories enables us to distinguish every possible vowel. Consonants are defined by the place and the manner of their production. The place, again, refers to the position of the tongue in the mouth and the overall form of the vocal tract. The vocal tract is used to block the air and make it flow in a specific way. The manner, on the other hand, describes the way the air is lead through the mouth or how it is blocked to produce a sound \citep{phonetics-video}. \review{finish consonant part}. The exact description of each phoneme will later become important when we talk about representing phonemes for a \ac{g2p} model in section \ref{phon-features}. \review{Check this again if this is needed.}

\subparagraph{Syllables} Phonemes, or letters, can be grouped into larger units called \textit{syllables}. Syllables can be an entire word or a part of a word. English syllables typically consist of a group of consonants followed by a group of vowels or a diphthong followed by a group of consonants again. These parts are called \textit{onset}, \textit{nuleus} and \textit{coda} respectively.  For every syllable in every language it is true that the nucleus cannot be empty. The onset and the coda can be empty. Other than that, syllables are organized very differently in different languages. \citep{Intro.2007}

\review{add explanation of categorization of consonants}

\review{add explanation of syllables, monophthongs, diphthong, suprasegmental they appear quite often in the lit (maybe make a glossary}

\section{Mappings of written and spoken language}
\label{writing-sys}
Unlike spoken language that was a part of human interaction all the time, writing systems only developed over time. There are different writing systems that developed in different places at different times. The structure of the spoken language, the cultural context or the tools that were at hand to write are a few of many factors that influenced the emergence of a specific writing system. In General, we can think of writing systems as mappings from spoken language to written language. The systems used to represent sounds in different languages do not uniquely map a letter to one specific phoneme. Most of the time, there is a standard pronunciation of each letter that is trained by reciting the alphabet. However, in reciting the alphabet there is a vowel added to the consonants in order to pronounce them more easily. These explanations make clear that the mapping of written text to spoken text in various languages is complex. When taking a step back, we can see that a single grapheme can represent either a phoneme, a syllables or words. Each mapping will be presented below: 

\begin{description}
\item[\textsc{Alphabet}] When a grapheme maps to a phoneme, we call this an alphabet. In German, for example, the writing system consists of the Latin alphabet. The Latin alphabet is used for many different languages in western Europe and those languages that were influence by colonisation. There are other alphabets like the Cyrillic or the Greek alphabet. Having an alphabet does not mean that each grapheme, or letter in this case, maps to exactly one phoneme. In fact, one grapheme can have many different realizations as example \ref{ex:latin-alpha} shows.
\begin{covsubexamples}[preamble={The examples show the different realizations of the English grapheme sequence `ough' \citep{phonetics-video}}]
\label{ex:latin-alpha}
\item tough \>\> [\textipa{t\textturnv f}]
\item cough \>\> [\textipa{k\textturnscripta f}]
\item though \>\> [\textipa{\dh\textschwa\textupsilon}]
\item through \>\> [\textipa{\texttheta ru:}]
\item bough \>\>  [\textipa{baU}] %baʊ
\item brought \>\> [\textipa{brO:t}] %brɔːt
\end{covsubexamples}

The above examples show that it is not possible to have a one-to-one mapping from one grapheme or a sequence of graphemes to one phoneme or a sequence of phonemes with in the English language. Let alone within all languages that use the Latin alphabet. In addition, alphabets typically have diacritic marks that can be used to extend the main letters. Just as with single graphemes, also diacritic marks cannot simply be mapped to a phoneme.

\item[\textsc{Abjad}] A special variant of an alphabet-language is abjad. Abjad represents only consonants and no vocals. Semitic languages like Hebrew or Arabic make use of abjad.

\begin{covsubexamples}[preamble={Hebrew examples that are first mapped to Latin alphabet then to phonemes.}]
\label{ex:abjad}
\item
\end{covsubexamples}
\item[\textsc{Syllabary}] In syllabaries, a grapheme represents a syllable instead of a single sound. Examples are the Japanese Hiragana and Katakana.
\item[\textsc{Logographic systems}] Logographic systems represent entire words or morphemes as graphemes. Chinese is an example for a logographic system. We cannot break down Chinese signs into single morphemes or letters. 
\end{description}

What all of these mappings have in common is that they are no reliable source of pronunciation \citep{Intro.2007}.

The history and development of writing systems is an entire independent study area. For this thesis it is mostly important to be aware of the independently developing systems. Not all scripts can be treated the same and this most certainly has implications on models to create phonetic transcription. 

An exception to the above explained characteristics of an alphabet are phonetic alphabets like the \ac{ipalpha} where each grapheme represents exactly one phone  \citep{writing-systems, Intro.2007}. More on this special alphabet will be explained in section \ref{transcb-conventions}.

Many of the pronunciation rules of a language are based on convention. Speakers of a language just \textit{know} how to pronounce a word. Still, there can arise heated debates about the correct pronunciation of certain words. \review{add some info here.} Apart from these conventions, spoken and written languages change differently over time. Spoken languages are typically more flexible and ready to change while their written representation often stays the same \citep{unicode-lingu}. This can lead to official governmental interventions like the German orthography reform of 1996 that intended to adapt the German spelling to represent the German pronunciation more adequately. Also, major inventions like printing machines gave rise to standardization of writing systems as reading and writing became more common.

\section{The corpus}
\label{corpus}
As mentioned int eh introduction, the basis of the data used in this thesis is a corpus provided by the SPUR lab at UZH. The corpus contains 100 languages which are proposed by \citet{Comrie&Dryer.2013}. This online book contains different chapters each of which shows a different linguistic feature including a map which shows the distribution of that feature over the world's languages. While the number of languages presented on the individual maps depends on the amount of research done in a specific area, the sum of all maps gives quite an impressive overview on the structure of nearly half of the world's languages. Out of the 2676 languages a sample of 100 languages was chosen. This sample does not contain too many languages from one area, neither does it contain too many languages from one family. Not considering the aforementioned criteria of maximizing genealogical and areal diversity can lead to misleading results. Figure \ref{fig:100lc} shows the distribution of the corpus on a world map. The different icons show the genus of the languages which is a classification of languages defined by the \ac{wals} team that maintains the language collection. The interactive map can be viewed online \citep{100LC.21.07.2021}. Table \ref{tab:100LC} in the appendix A shows all languages that are in the 100 language corpus. None of the text samples are provided by \ac{wals}. The entire corpus is provided by the SPUR team that collected the corpus over the last few years and is continuously working on and with it.

%\fig{#1: filename}{#2: label}{#3: long caption}{#4: width}{#5: short caption}
\fig{images/100sample.png}{fig:100lc}{WALS - 100 Language Sample}{\textwidth}{100 Language Sample}









\section{Corpus phonetics}
Due to recent technological advancement it has become possible to store large digital collections of speech recordings and their aligned transcriptions. These possibilities gave rise to a wider acknowledgement of corpus phonetics. Corpus phonetics deals with an abundance of linguistic variation. In addition to language, style or vocabulary variation, there are differences in dialect and idiolect, physiological state of the speakers and their attitude \citep{Liberman.2019, Chodroff.19.07.2019}. Many methods and tools used in corpus phonetics are based on \ac{asr} algorithms or simple programming \citep{Chodroff.19.07.2019}.

A way to analyse or use phonetic corpora is to use phonetic features to represent each phoneme. These features are a list of properties that are overlapping with the phonetic description of each phoneme. It is a list that can minimally be used to describe the phonemes. 


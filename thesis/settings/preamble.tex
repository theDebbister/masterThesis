\documentclass[a4paper, 12pt,numbers=noenddot]{scrreprt}

\usepackage[utf8x, utf8]{inputenc}
\usepackage[scaled]{helvet}
\usepackage[ngerman,english]{babel} %Trennungen, Schriftsatz; Neue deutsche Rechtschreibung
\usepackage[square]{natbib}  % more options for citations, e.g. here: square brackets as ``parentheses''
\renewcommand{\bibsection}{}

\usepackage[pdftex]{graphicx}

%\usepackage{fontspec}


\usepackage[T1]{fontenc}
\usepackage{tipa}
\usepackage{vowel}
\let\ipa\textipa

\newcommand{\BlankCell}{}
%\setmainfont{Doulos SIL}

\usepackage{qtree}
\usepackage{wrapfig}
\usepackage[table]{xcolor}
\usepackage{tabularx}
%\usepackage{caption}
\usepackage{pdfpages}
\usepackage[strict]{changepage}
\usepackage{pdflscape}
\usepackage{cjhebrew}
\usepackage{CJKutf8}
\usepackage{enumitem}
\usepackage{amsmath}

\usepackage{booktabs} % mehr Optionen zur Gestaltung von Tabellen
\usepackage{covington} % 'examples' Umgebung fuer linguistische Beispiele
\usepackage[l3]{csvsimple}
\usepackage{longtable}
\usepackage{multicol}
\usepackage{multirow}
\usepackage[printonlyused]{acronym}
\usepackage{xurl}

\usepackage{auncial}
%\usepackage[B1]{fontenc}
\defcitealias{JIPA2010}{Cambridge University Press, 2010}
\defcitealias{Unicode.27.08.2021}{Unicode Inc., 2021}

% define toc formatting
\usepackage[titles]{tocloft}
\setlength{\cftsubsecindent}{3em}
\setlength{\cftsubsecnumwidth}{3.3em}
\setlength{\cftsubsubsecindent}{4.5em}
\setlength{\cftsubsubsecnumwidth}{4em}
\setlength{\examplenumbersep}{0.5em}


% figure numbering
\newcounter{myfigure}
\renewcommand{\thefigure}{\arabic{myfigure}}
\newcounter{mytable}
\renewcommand{\thetable}{\arabic{mytable}}
\usepackage{makeidx} \makeindex
\makeglossary 


% Formatierung von Referenzen
\usepackage[hyperfootnotes=false]{hyperref}
\hypersetup{%
  pdfauthor={Deborah Jakobi},
  pdftitle={Multilingual Grapheme-to-Phoneme Conversion},
  pdfsubject={},
  pdfkeywords={},
  pdfborder=000
}
% so werden Hyperlinks nicht gefärbt


% Definition vom Seitenlayout
\setlength{\topmargin}{-1.2cm}
\setlength{\oddsidemargin}{0.5cm} 
\setlength{\evensidemargin}{0.5cm}

\setlength{\textheight}{24.5cm} 
\setlength{\textwidth}{15cm}

\setlength{\footskip}{1.2cm} 
\setlength{\footnotesep}{0.4cm}

% Definition vom Header und Footer im Seitenlayout
\usepackage{fancyhdr} 
\pagestyle{fancy} 
\fancyhf{}

% Linien nach dem Header und vor dem Footer
%\renewcommand{\footrulewidth}{0.4pt}
\renewcommand{\headrulewidth}{0.4pt}

\fancyhead[L]{\footnotesize{\leftmark}}
\fancyhead[R]{}
\fancyfoot[C]{\footnotesize{\thepage}}

% Plain Pagestyle für Kapitelanfangsseite
\fancypagestyle{plain}{%
  \fancyhf{} 
  \renewcommand{\headrulewidth}{0pt}
  \fancyfoot[C]{\footnotesize{\thepage}}
}


% Neues Kapitel Makro, damit die Variablen korrekt abgefüllt werden
\newcommand{\newchap}[1]{
	\chapter{#1}
	\markboth {Chapter \thechapter.  {#1}}{Chapter \thechapter.  {#1}}
}

\newcommand{\review}[1]{\textcolor{red}{#1}}

% Footnote numbering
\newcounter{myfootnote}[chapter]
\renewcommand{\thefootnote}{\themyfootnote}
\newcommand{\myfootnote}[1]{\stepcounter{myfootnote}\footnote{#1}}

% command to import a figure
\newcommand{\fig}[5]{
  \begin{figure}[h]
    \begin{center}
      \includegraphics[width=#4]{#1}
    \end{center}
    \stepcounter{myfigure}
    \caption[#5]{#3}
    \label{#2}
  \end{figure}
}

\newcommand{\figtable}[4]{
  \begin{figure}[h]
    \begin{center}
      {  
	\footnotesize
	\sffamily
	#3
      }
    \end{center}
    \stepcounter{myfigure}
    \caption[#4]{#2}
    \label{#1}
  \end{figure}
}

\newcommand{\tab}[4]{
  \begin{table}[h!]
    \begin{center}
      {  
	\footnotesize
	%\sffamily
	\renewcommand{\arraystretch}{1.4}
	#3
      }
    \end{center}
    \stepcounter{mytable}
    \caption[#4]{#2}
    \label{#1}
  \end{table}
}

% command to refere to a figure
\newcommand{\reffig}[1]{Figure \ref{#1}}

% examples as floating environment
\usepackage{float}
\newfloat{example}{tbp}{loe}[chapter]
\floatname{example}{Example}

%\floatplacement{figure}{htbp}
%\floatplacement{table}{htbp}
\floatplacement{fexample}{htbp}

% Definition des Nummerierungslevel
\setcounter{secnumdepth}{4} 
\setcounter{tocdepth}{4} 
\setcounter{lofdepth}{1} 

% Definition von Paragraphen
\parskip=0.3cm
\parindent=0cm

% Definition vom Zeilenabstand
\usepackage{setspace} % Zeilenabstand
\onehalfspacing %\doublespace or \singlespace

% Befehle für Anführungszeichen
\usepackage{xspace} % Leerschlag nach Anführungszeichen
\newcommand{\qr}{\grqq\xspace}
\newcommand{\qrs}{\grqq\ }
\newcommand{\ql}{\glqq}

% Formatierungsbefehle zum Zitieren
\newcommand{\page}[1]{p.~#1}
\newcommand{\pagef}[1]{p.~#1f.}
\newcommand{\pageff}[1]{p.~#1ff.}
\newcommand{\pages}[2]{p.~#1--#2}

\newcommand{\etiq}[1]{\textbf{#1}}
\newcommand{\token}[1]{\textit{#1}}

\usepackage{supertabular}